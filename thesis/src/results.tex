\chapter{Results}
\label{ch:results}

\usepackage{csvsimple}
%In this chapter, you want to show that your implementation meets the objectives.
%Start by briefly describing the type of results you have collected and how these results relate to the objectives.
%
%\section{Evaluation setup}
%
%Precisely describe your measurement/evaluation setup in a top-down manner.
%For example:
%\begin{itemize}
%  \item If you used a development platform, which one and how was it configured?
%  \item Which tools did you use?
%  \item What input data set / program / signals / \ldots{} did you use?
%    If your data set is very large, you should fully define and describe it in an appendix chapter and refer to that definition here using simple labels.
%  \item If your implementation is configurable, which configuration(s) did you evaluate?
%\end{itemize}
%
%The information given here (with references to external work) should be sufficient to reproduce the setup you used for your measurements.
%
%\section{UCR Dataset Archive Results}
%\include{./data/csv_table.tex}
%
%
%Structure the rest of this chapter into sections.
%For example, each section could discuss a different figure of merit or a different part of your implementation.
%As usual, discuss structure and content in advance with your advisors.
%
%\section{Comparison to related work}
%
%Compare your results to those achieved by others working on similar problems.
%This can be done in a separate section or directly in the sections above.


\subchapter{Speech Commands - Edge Implementations}

\begin{table}
\begin{tabular}{||c|c|c|c|c|c|r|r||}
    \hline
    Reference & Model & TestAC & Memory & Ops & Year & Citation & Reproducible? \\

    \hline\hline
    \cite{Zhang2017}   &        DS-CNN & 94.4\% & 38.6 kB & 5.4M & 2017 & 483 &  \href{https://github.com/ARM-software/ML-KWS-for-MCU}{Yes} \\
    \cite{Andrade2018} &       Att-RNN & 96.9\% &  202 kB &   ?? & 2018 & 113 &  No \\
    \cite{Tang2018}    &       ResNet8 & 94.1\% &  110 kB &  30M & 2018 & 241 &  \href{https://github.com/castorini/honk/}{Yes} \\
    \cite{Tang2018}    & ResNet-Narrow & 90.1\% & 19.9 kB & 5.7M & 2018 & 241 &  \href{https://github.com/castorini/honk/}{Yes} \\
    \cite{Jansson2018} &    ConvNetRaw & 89.4\% &  700 kB &   5M & 2018 &  19 &  No \\
    \textbf{Ours}      &     NanoHydra & 88.7\% & 56.1 kB & 2.6M & 2024 & n/a &  Yes \\
    \hline
\end{tabular}
\end{table}

%\section{Current limitations}
%
%Demonstrating that your implementation meets the objectives is usually done by showing a lower bound on a given figure of merit.
%Conversely, you should also illuminate the limitations of your implementation by showing upper bounds on these (or other appropriate) figures of merit.
%Critically examining your ideas and their implementation trying to find their limits is also part of your work!
