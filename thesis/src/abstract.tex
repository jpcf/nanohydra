\chapter*{Abstract}

This work aims to implement Hydra, a time-series classification algorithm, in the GAP9 microcontroller, and benchmark its suitability
as a lightweight, energy-efficient model for edge computation. Hydra is proposed as a target-agnostic time-series classification algorithm, that
shows close to state-of-the-art accuracy over the UCR Dataset Archive (a time-series classification benchmark), while at the same time training and 
doing inference at a fraction of the time that other similarly accurate algorithms need. This makes it a natural candidate for an embedded implementation.
We explain the changes made to Hydra to make it even more suitable to the memory, performance and energy constraints of an embedded system  while minimizing
the possible loss in accuracy that results from that process. 

To this new and optimized model we call \emph{Nano}Hydra, and we benchmark its accuracy against 
the original results of Hydra, and we use a specific dataset from the UCR Archive, the ECG5000, to demonstrate its fitness as an energy-efficient time-series classification
algorithm, comparing its different topologies, the trade-offs between performance and design complexity, and how it compares to similar works in literature.
NanoHydra maintains a good level of accuracy across the UCR Archive Dataset, and achieves state-of-the-art accuracy and embedded performance in the ECG5000 dataset. 
Furthermore, it is possible to train it in just under 5 seconds using a multicore CPU, without needing GPUs to accelerate training. 

Lastly, we enhance Hydra/NanoHydra by proposing a \emph{multichannel} architecture, and test it with the Google Speech Commands dataset, one of \emph{de facto} standards for KWS.

In conclusion, NanoHydra is a fast and accurate time-series classification algorithm suitable as a potential foundational model for time-series classification on edge devices, which not only provides lightweight inference, but also
extremely fast training times.


%The abstract summarizes what this report is about.
%It focusses on the big picture and does not go into details.
%You should write concisely about the following points:
%
%\begin{itemize}
%  \item Describe the \textbf{background} of your project: what is the motivation for your project and why is it important?
%  \item Describe the \textbf{objectives} of your project.
%  \item Describe the \textbf{problems} that must be addressed to achieve the objectives---why are these problems difficult?
%  \item Describe your \textbf{approach} and \textbf{methods}.
%  \item Summarize the most important \textbf{results}.
%  \item State the main \textbf{conclusion} and its significance.
%\end{itemize}
%
%The abstract typically takes half a page and should not be longer than one full page.
%Try to write a draft of the abstract early on to have a good idea of your project, but revise the abstract as the project progresses.
%Write the final version of the abstract once the report is otherwise complete.
%
%The remainder of this document contains an example on structure and content of the report.
%This template is meant to guide you and not to force you into a certain structure---just make sure you and your advisors agree on content and structure of the report \emph{before} you start writing it.
%\Cref{app:topic-specific_guidelines} gives more specific guidelines for some major project areas (e.g., hardware designs).
%If you are new to \LaTeX{} or want to learn some best practices, you should also check the short \LaTeX{} guide in \cref{app:LaTeX_guide}.
