\chapter*{Abstract}

This work aims to implement Hydra, a time-series classification algorithm, in the GAP9 microcontroller, and benchmark its suitability
as a lightweight, energy-efficient model for edge computation. Hydra is proposed as a target-agnostic time-series classification algorithm, that
shows close to state-of-the-art accuracy over the UCR Dataset Archive (a time-series classification benchmark), while at the same time training and 
doing inference at a fraction of the time that other similarly accurate algorithms need. This makes it a natural candidate for an embedded implementation.
We explain the changes made to Hydra to make it even more suitable to the memory, performance and energy constraints of an embedded system  while minimizing
the possible loss in accuracy that results from that process. 

To this new and optimized model we call \emph{Nano}Hydra, and we benchmark its accuracy against 
the original results of Hydra, and we use a specific dataset from the UCR Archive, the ECG5000, to demonstrate its fitness as an energy-efficient time-series classification
algorithm, comparing its different topologies, the trade-offs between performance and design complexity, and how it compares to similar works in literature.
NanoHydra maintains a good level of accuracy across the UCR Archive Dataset, and achieves state-of-the-art accuracy and embedded performance in the ECG5000 dataset. 
Furthermore, it is possible to train it in just under 5 seconds using a multicore CPU, without needing GPUs to accelerate training. 

Lastly, we enhance Hydra/NanoHydra by proposing a \emph{multichannel} architecture, and test it with the Google Speech Commands dataset, one of \emph{de facto} standards for KWS.

In conclusion, NanoHydra is a fast and accurate time-series classification algorithm suitable as a potential foundational model for time-series classification on edge devices, which not only provides lightweight inference, but also
extremely fast training times.
