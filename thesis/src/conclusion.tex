\chapter{Conclusion and Future Work}
\label{ch:conclusion}


NanoHydra proves to be a suitable time-series classification algorithm for generic time-series classification on the edge. It maintains a good overall level of accuracy
over the entire UCR Archive, with deviations of up to -20\% in some corner cases, but mostly equaling or even exceeding the accuracy of Hydra by a difference of almost 10\%. All of this, 
while being a smaller model than Hydra, with a fully integer inference pipeline and tailored for the memory and performance constraints of edge-devices. The GAP9 implementation
makes use of the low-power capabilities of the MCU, and leverages its Vectorization and Parallelization capabilities, proving their suitability for real-world tasks with the 
performance on the ECG5000 dataset, where SOTA accuracy is maintained while having inference times lower than 1 milliseconds, in some configurations. 

The GAP9 implementation is fully customizable, by easily toggling compiler switches for parallelization and vectorization levels, allowing the user to effortlessly achieve the energy/inference time trade-off level
that best suits his requirements, with no changes required in the code. This makes NanoHydra and the delivered libraries that accompany this work as a good general model for a variety of time-series classification
problems.

Lastly, the proposed multichannel architecture of Hydra/NanoHydra was tested against the Google Speech Commands dataset, and although the accuracy is lower than SOTA results, it constitutes
an important lead for \textbf{future work}, where a different multichannel receptive field can better extract patterns \emph{across} MFCC channels, rather than in each channel separately.
